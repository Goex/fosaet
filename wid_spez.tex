% coding:utf-8

%FOSAET, a LaTeX-Code for a electrical summary of basic electronics
%Copyright (C) 2013, Daniel Winz, Ervin Mazlagic

%This program is free software; you can redistribute it and/or
%modify it under the terms of the GNU General Public License
%as published by the Free Software Foundation; either version 2
%of the License, or (at your option) any later version.

%This program is distributed in the hope that it will be useful,
%but WITHOUT ANY WARRANTY; without even the implied warranty of
%MERCHANTABILITY or FITNESS FOR A PARTICULAR PURPOSE.  See the
%GNU General Public License for more details.
%----------------------------------------

\subsection{Widerstand einer Leitung}
\[ R = \frac{\rho \cdot \ell}{A} \]
\begin{tabular}{lp{0.8\textwidth}}
$\rho$&Spezifischer Widerstand\\
&(Achtung! liegt meist nicht in SI-Einheiten vor)\\
$\ell$&Länge\\
   $A$&Fläche
\end{tabular}

\subsubsection{Spezifischer Widerstand gängiger Materialien}
\begin{table}[h!]
\begin{tabular}{lr}
  Silber    & $1.63 \cdot 10^{-2} \frac{\Omega \cdot mm^2}{m}$ \\
  Kupfer    & $1.73 \cdot 10^{-2} \frac{\Omega \cdot mm^2}{m}$ \\
  Gold      & $2.21 \cdot 10^{-2} \frac{\Omega \cdot mm^2}{m}$ \\
  Aluminium & $2.63 \cdot 10^{-2} \frac{\Omega \cdot mm^2}{m}$ \\
  Messing   & $7.52 \cdot 10^{-2} \frac{\Omega \cdot mm^2}{m}$ \\
  Manganin  & $0.435 \frac{\Omega \cdot mm^2}{m}$ \\
\end{tabular}
\label{tab_spezwid}
\caption{Werte aus den Unterrichtsunterlagen}
\end{table}