% coding:utf-8

%FOSAET, a LaTeX-Code for a electrical summary of basic electronics
%Copyright (C) 2013, Daniel Winz, Ervin Mazlagic

%This program is free software; you can redistribute it and/or
%modify it under the terms of the GNU General Public License
%as published by the Free Software Foundation; either version 2
%of the License, or (at your option) any later version.

%This program is distributed in the hope that it will be useful,
%but WITHOUT ANY WARRANTY; without even the implied warranty of
%MERCHANTABILITY or FITNESS FOR A PARTICULAR PURPOSE.  See the
%GNU General Public License for more details.
%----------------------------------------

\section{Werte von Wechselgrössen}

\subsection{Gleichwert (Mittelwert)}
\[ \overline{U} = \frac{1}{T} \int_0^T u(t) ~ dt \]

\subsection{Gleichrichtwert}
Der Gleichrichtwert bezeichnet den Mittelwert der gleichgereichteten 
Wechselgrösse
\[ \overline{|U|} = \frac{1}{T} \int_0^T |u(t)| ~ dt \]

\subsection{Effektivwert (Quadratischer Mittelwert)}
Der Effektivwert entspricht dem Gleichspannungsewrt, der in einem Widerstand 
die gleiche Leistung umsetzt, wie die Wechselgrösse. 
\[ U_{eff} = \sqrt{\frac{1}{T} \int_0^T u(t)^2 ~ dt} \]

\subsubsection{Effektivwert bei verschiedenen Abschnitten}
\[ U_{eff} = \sqrt{\frac{{U_{eff_1}}^2 \cdot \Delta t_1 
+ {U_{eff_2}}^2 \cdot \Delta t_2 + \dots}{T}} \]

\subsubsection{Mischgrössen}
\[ U_{eff} = \sqrt{{U_{DC}}^2 + {U_{eff_{AC}}}^2} \]

\subsubsection{Effektivwert von häufigen Signalformen}
Sinus
\[ U_{eff} = \frac{\hat{u}}{\sqrt{2}} \]
%
Dreieck
\[ U_{eff} = \frac{\hat{u}}{\sqrt{3}} \]
%
Rechteck
\[ U_{eff} = \hat{u} \]
%
Alle Signale müssen symmetrisch sein. 

